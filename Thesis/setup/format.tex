% !Mode:: "TeX:UTF-8"
%  Authors: 张井   Jing Zhang: prayever@gmail.com     天津大学2010级管理与经济学部信息管理与信息系统专业硕士生
%           余蓝涛 Lantao Yu: lantaoyu1991@gmail.com  天津大学2008级精密仪器与光电子工程学院测控技术与仪器专业本科生


%%%%%%%%%%%%%%%%% Fonts Definition and Basics %%%%%%%%%%%%%%%%%
%\newcommand{\song}{\CJKfamily{song}}    % 宋体
%\newcommand{\fs}{\CJKfamily{fs}}        % 仿宋体
%\newcommand{\kai}{\CJKfamily{kai}}      % 楷体
%\newcommand{\hei}{\CJKfamily{hei}}      % 黑体
%\newcommand{\li}{\CJKfamily{li}}        % 隶书
\newcommand{\song}{\songti}    % 宋体
\newcommand{\fs}{\fangsong}        % 仿宋体
\newcommand{\kai}{\kaishu}      % 楷体
\newcommand{\hei}{\heiti}      % 黑体
\newcommand{\li}{\lishu}        % 隶书
\newcommand{\yihao}{\fontsize{26pt}{26pt}\selectfont}       % 一号, 单倍行距
\newcommand{\xiaoyi}{\fontsize{24pt}{24pt}\selectfont}      % 小一, 单倍行距
\newcommand{\erhao}{\fontsize{22pt}{1.25\baselineskip}\selectfont}       % 二号, 1.25倍行距
\newcommand{\xiaoer}{\fontsize{18pt}{18pt}\selectfont}      % 小二, 单倍行距
\newcommand{\sanhao}{\fontsize{16pt}{16pt}\selectfont}      % 三号, 单倍行距
\newcommand{\sanhaoup}{\fontsize{16pt}{20pt}\selectfont}    % 三号, 1.25倍行距
\newcommand{\xiaosan}{\fontsize{15pt}{15pt}\selectfont}     % 小三, 单倍行距
\newcommand{\sihao}{\fontsize{14pt}{14pt}\selectfont}       % 四号, 单倍行距
\newcommand{\xiaosi}{\fontsize{12pt}{12pt}\selectfont}      % 小四, 单倍行距
\newcommand{\xiaosiup}{\fontsize{12pt}{15pt}\selectfont}      % 小四, 单倍行距
\newcommand{\wuhao}{\fontsize{10.5pt}{10.5pt}\selectfont}   % 五号, 单倍行距
\newcommand{\xiaowu}{\fontsize{9pt}{9pt}\selectfont}        % 小五, 单倍行距

%\CJKtilde  % 重新定义了波浪符~的意义
% JUST DON'T USE CJK
% 使用 ctexbook 之后已无必要
\newcommand\prechaptername{第}
\newcommand\postchaptername{章}

\punctstyle{hangmobanjiao}             % 调整中文字符的表示,行内占一个字符宽度,行尾占半个字符宽度

% 调整罗列环境的布局
\setitemize{leftmargin=3em,itemsep=0em,partopsep=0em,parsep=0em,topsep=-0em}
\setenumerate{leftmargin=3em,itemsep=0em,partopsep=0em,parsep=0em,topsep=0em}

% 避免宏包 hyperref 和 arydshln 不兼容带来的目录链接失效的问题。
\def\temp{\relax}
\let\temp\addcontentsline
\gdef\addcontentsline{\phantomsection\temp}

% 自定义项目列表标签及格式 \begin{publist} 列表项 \end{publist}
\newcounter{pubctr} %自定义新计数器
\newenvironment{publist}{%%%%%定义新环境

  \begin{list}{[\arabic{pubctr}]} %%标签格式
    {
      \usecounter{pubctr}
      \setlength{\leftmargin}{2.5em}   % 左边界 \leftmargin =\itemindent + \labelwidth + \labelsep
      \setlength{\itemindent}{0em}     % 标号缩进量
      \setlength{\labelsep}{1em}       % 标号和列表项之间的距离,默认0.5em
      \setlength{\rightmargin}{0em}    % 右边界
      \setlength{\topsep}{0ex}         % 列表到上下文的垂直距离
      \setlength{\parsep}{0ex}         % 段落间距
      \setlength{\itemsep}{0ex}        % 标签间距
      \setlength{\listparindent}{0pt}  % 段落缩进量
  }}
{\end{list}}

\makeatletter
\renewcommand\normalsize{
  \@setfontsize\normalsize{12pt}{12pt} % 小四对应 12 pt
  \setlength\abovedisplayskip{4pt}
  \setlength\abovedisplayshortskip{4pt}
  \setlength\belowdisplayskip{\abovedisplayskip}
  \setlength\belowdisplayshortskip{\abovedisplayshortskip}
  \let\@listi\@listI
}

\def\defaultfont{\renewcommand{\baselinestretch}{1.63}\normalsize\selectfont} % 设置行距
%
\renewcommand{\CJKglue}{\hskip -0.1 pt plus 0.08\baselineskip} % 控制字间距,使每行 34 个汉字
\makeatother

%%%%%%%%%%%%% Contents %%%%%%%%%%%%%%%%%
\renewcommand{\contentsname}{\centering \xiaosan \heiti 目\quad 录}
\setcounter{tocdepth}{2} % 控制目录深度

\titlecontents{chapter}[2em]
  {\vspace{.25\baselineskip}\xiaosi\songti}
  {\songti\thecontentslabel\quad \songti}
  {}
  {\hspace{.3em}\titlerule*{.}\contentspage}

\titlecontents{section}[3em]
  {\vspace{.25\baselineskip}\xiaosi\songti}
  {\songti\thecontentslabel\quad \songti}
  {}
  {\hspace{.5em}\titlerule*{.}\contentspage}
%
\titlecontents{subsection}[4em]
  {\vspace{.25\baselineskip}\xiaosi\songti}
  {\songti\thecontentslabel\quad \songti}
  {}
  {\hspace{.5em}\titlerule*{.}\contentspage}

%%%%%%%%%% Chapter and Section %%%%%%%%%%%%%
\setcounter{secnumdepth}{4}
\setlength{\parindent}{1em}
\renewcommand{\chaptername}{\prechaptername\CJKnumber{\thechapter}\postchaptername\hspace{-1em}}

% Chapter 第一章 绪论 [xiaosan, heiti]
\titleformat{\chapter}[block]
  {\centering\xiaosan\heiti} % format
  {} % label
  {0pt} % sep
  {} % before-code
\titlespacing{\chapter}
  {0pt} % 左边距
  {0pt} % 上面的空白
  {20pt} % 下面的空白

% Section
\titleformat{\section}{\sihao\heiti}{\thesection}{1em}{}
\titlespacing{\section}{0pt}{18pt}{18pt}

% Subsection
\titleformat{\subsection}{\xiaosi\heiti}{\thesubsection}{1em}{}
\titlespacing{\subsection}{0pt}{12pt}{12pt}

% SubsubSection
\titleformat{\subsubsection}{\xiaosi\heiti}{\thesubsubsection}{1em}{}
\titlespacing{\subsubsection}{0pt}{10pt}{10pt}

%%%%%%%%%% Table, Figure and Equation %%%%%%%%%%%%%%%%%
\renewcommand{\tablename}{表}                                     % 插表题头
\renewcommand{\figurename}{图}                                    % 插图题头
\renewcommand{\thefigure}{\arabic{chapter}.\arabic{figure}}       % 使图编号为 7.1 的格式 %\protect{~}
\renewcommand{\thesubfigure}{\alph{subfigure}}                   % 使子图编号为 subfig 自动就是 (a) 的格式 这里保留只是作为参考
\renewcommand{\thesubtable}{(\alph{subtable})}                    % 使子表编号为 (a) 的格式
\renewcommand{\thetable}{\arabic{chapter}.\arabic{table}}         % 使表编号为 7.1 的格式
\renewcommand{\theequation}{\arabic{chapter}.\arabic{equation}}   % 使公式编号为 7.1 的格式
\newcommand{\ud}{\mathrm{d}}

%%%%%% 定制浮动图形和表格标题样式 %%%%%%
\makeatletter
\long\def\@makecaption#1#2{
  \vskip\abovecaptionskip
  \sbox\@tempboxa{\centering\wuhao\song{#1\qquad #2} }
  \ifdim \wd\@tempboxa >\hsize
    \centering\wuhao\song{#1\qquad #2} \par
  \else
    \global \@minipagefalse
    \hb@xt@\hsize{\hfil\box\@tempboxa\hfil}
  \fi
\vskip\belowcaptionskip}
\makeatother
\captiondelim{~~~~} %用来控制longtable表头分隔符

%%%%%%%%%% Theorem Environment %%%%%%%%%%%%%%%%%
\theoremstyle{plain}
\theorembodyfont{\song\rmfamily}
\theoremheaderfont{\hei\rmfamily}
\newtheorem{theorem}{定理~}[chapter]
\newtheorem{lemma}{引理~}[chapter]
\newtheorem{axiom}{公理~}[chapter]
\newtheorem{proposition}{命题~}[chapter]
\newtheorem{prop}{性质~}[chapter]
\newtheorem{corollary}{推论~}[chapter]
\newtheorem{definition}{定义~}[chapter]
\newtheorem{conjecture}{猜想~}[chapter]
\newtheorem{example}{例~}[chapter]
\newtheorem{remark}{注~}[chapter]
%\newtheorem{algorithm}{算法~}[chapter]
\newenvironment{proof}{\noindent{\hei 证明:}}{\hfill $ \square $ \vskip 4mm}
\theoremsymbol{$\square$}

%%%%%%%%%% Page: number, header and footer  %%%%%%%%%%%%%%%%%

%\frontmatter 或 \pagenumbering{roman}
%\mainmatter 或 \pagenumbering{arabic}

%%%%%%%%%%% Code: Listings from MCM Template %%%%%%%%%%%%

\definecolor{grey}{rgb}{0.8,0.8,0.8}
\definecolor{darkgreen}{rgb}{0,0.3,0}
\definecolor{darkblue}{rgb}{0,0,0.3}
\def\lstbasicfont{\fontfamily{pcr}\selectfont\footnotesize}
\lstset{%
  % indexing
  % numbers=left,
  % numberstyle=\small,%
  % character display
  showstringspaces=false,
  showspaces=false,%
  tabsize=4,%
  % style
  frame=lines,%
  basicstyle={\footnotesize\lstbasicfont},%
  keywordstyle=\color{darkblue}\bfseries,%
  identifierstyle=,%
  commentstyle=\color{darkgreen},%\itshape,%
  stringstyle=\color{black}%
}
\lstloadlanguages{C,C++,Java,Matlab,Mathematica,Python}

%%%%%%%%%%%% References %%%%%%%%%%%%%%%%%
\renewcommand{\bibname}{参考文献}

% 重定义参考文献样式,来自thu
\makeatletter
\renewenvironment{thebibliography}[1]{
  \addcontentsline{toc}{chapter}{参考文献}
  \chapter*{\bibname}
  \wuhao
  \list{\@biblabel{\@arabic\c@enumiv}}
  {\renewcommand{\makelabel}[1]{##1\hfill}
    \settowidth\labelwidth{0 cm}
    \setlength{\labelsep}{0pt}
    \setlength{\itemindent}{0pt}
    \setlength{\leftmargin}{\labelwidth+\labelsep}
    \addtolength{\itemsep}{-0.7em}
    \usecounter{enumiv}
    \let\p@enumiv\@empty
  \renewcommand\theenumiv{\@arabic\c@enumiv}}
  \sloppy\frenchspacing
  \clubpenalty4000
  \@clubpenalty \clubpenalty
  \widowpenalty4000
  \interlinepenalty4000
\sfcode`\.\@m}
{\def\@noitemerr
  {\@latex@warning{Empty `thebibliography' environment}}
\endlist\frenchspacing}
\makeatother

\addtolength{\bibsep}{-0.5em}     % 缩小参考文献间的垂直间距
\setlength{\bibhang}{2em}         % 每个条目自第二行起缩进的距离

% 参考文献引用作为上标出现
% \newcommand{\citeup}[1]{\textsuperscript{\cite{#1}}}
\makeatletter
\def\@cite#1#2{\textsuperscript{[{#1\if@tempswa , #2\fi}]}}
\makeatother
%% 引用格式
\bibpunct{[}{]}{,}{s}{}{,}

%%%%%%%%%%%% Cover %%%%%%%%%%%%%%%%%
% 封面、摘要、版权、致谢格式定义
\makeatletter
\def\ctitle#1{\def\@ctitle{#1}}\def\@ctitle{}
\def\cdegree#1{\def\@cdegree{#1}}\def\@cdegree{}
\def\caffil#1{\def\@caffil{#1}}\def\@caffil{}
\def\csubject#1{\def\@csubject{#1}}\def\@csubject{}
\def\cgrade#1{\def\@cgrade{#1}}\def\@cgrade{}
\def\cauthor#1{\def\@cauthor{#1}}\def\@cauthor{}
\def\cnumber#1{\def\@cnumber{#1}}\def\@cnumber{}
\def\csupervisor#1{\def\@csupervisor{#1}}\def\@csupervisor{}
\def\crank#1{\def\@crank{#1}}\def\@crank{}
\def\cdeclaration#1{\def\@cdeclaration{#1}}\def\@cdeclaration{}
\def\cdate#1{\def\@cdate{#1}}\def\@cdate{}
\long\def\cabstract#1{\long\def\@cabstract{#1}}\long\def\@cabstract{}
\long\def\eabstract#1{\long\def\@eabstract{#1}}\long\def\@eabstract{}
\def\ckeywords#1{\def\@ckeywords{#1}}\def\@ckeywords{}
\def\ekeywords#1{\def\@ekeywords{#1}}\def\@ekeywords{}
\def\cheading#1{\def\@cheading{#1}}\def\@cheading{}


\pagestyle{fancy}
\fancyhf{}
% \fancyhead[C]{\song\wuhao \@cheading}  % 页眉显示天津大学 20XX 届本科生毕业论文
% 定义页眉和页脚的内容
\fancyhead{}  % 清除默认页眉设置
\fancyhead[L]{\song\wuhao 天津理工大学硕士学位}  % 左侧页眉
\fancyhead[R]{\song\wuhao \leftmark}  % 右侧页眉,自动显示当前章节标题
\fancyfoot[C]{\song\xiaowu ~\thepage~}
\newlength{\@title@width}



% 定义封面
\def\makecover{
  %\cleardoublepage%
  \phantomsection
  \pdfbookmark[-1]{\@ctitle}{ctitle}

  \begin{titlepage}
    \begin{center}
        \song\xiaosi 中图分类号: \hfill \song\xiaosi 论文编号: \\
    \song\xiaosi 学科分类号: \hfill \song\xiaosi 密\qquad 级:

    
    % \vspace{21.6pt} %LaTeX 中的 1 磅等于约 0.3528 毫米,所以 19.2 磅约等于 6.77 毫米。
    \quad \\
    \textbf{\song\sanhao 天津理工大学研究生学位论文}
    \quad \\
    \quad \\
    \quad \\
    \quad \\
    \quad \\
    % \vspace{64.8pt} % 调整行距
    \textbf{\song\yihao 基于XXXXXXX的技术研究}

    % \vspace{12pt} % 调整行距
    \quad \\
    \textbf{\song\sanhao (申请硕士学位)}
    \quad \\
    \quad \\
    \quad \\
    \quad \\
    \quad \\
    \quad \\
    \quad \\
    \quad \\
    \quad \\
    \quad \\
    \quad \\
    \quad \\
    \quad \\
    \quad \\
    \quad \\
    \quad \\
    \quad \\
    % \vspace{200pt} % 调整行距
    \begin{spacing}{2}
    \song\sihao 一级学科:\textbf{\song\sihao 四号宋体加粗居中} \\
    \song\sihao 学科专业:\textbf{\song\sihao 四号宋体加粗居中} \\
    \song\sihao 研究方向:\textbf{\song\sihao 四号宋体加粗居中} \\
    \song\sihao 作者姓名:\textbf{\song\sihao 四号宋体加粗居中} \\
    \song\sihao 指导教师:\textbf{\song\sihao 四号宋体加粗居中}
    \end{spacing}
    % \song\sihao 一级学科:\textbf{\song\sihao 四号宋体加粗居中}
    
    % \vspace{10pt} % 调整行距
    % \song\sihao 学科专业:\textbf{\song\sihao 四号宋体加粗居中}
    
    % \vspace{10pt} % 调整行距
    % \song\sihao 研究方向:\textbf{\song\sihao 四号宋体加粗居中}
    
    % \vspace{10pt} % 调整行距
    % \song\sihao 作者姓名:\textbf{\song\sihao 四号宋体加粗居中}
    
    % \vspace{10pt} % 调整行距
    % \song\sihao 指导教师:\textbf{\song\sihao 四号宋体加粗居中}

    \vspace{50pt} % 调整行距
    \textbf{\song\sihao 2024年10月}
    \end{center}
  \end{titlepage}

  %%%%%%%%%%%%%%%%%%%   独创性声明  %%%%%%%%%%%%%%%%%%%%%%%
  \vspace{27.5pt}
  \markboth{独创性声明}{独创性声明}
  \pdfbookmark[0]{独创性声明}{cdeclaration}
  \chapter*{\textbf{\centering\erhao\songti 独创性声明}}
  \vspace{27.5pt}
  \sanhaoup % 三号字体 1.25 倍行距
  \@cdeclaration 
  \vspace{40pt} % 两行 三号字体 1.25 倍行距

  \begin{center}
    论文作者签名:
  \end{center}

  \hfill  年 \hspace{1cm} 月 \hspace{1cm} 日

  \vspace{40pt} % 两行 三号字体 1.25 倍行距
  本人声明:本毕业设计(论文)是本人指导学生完成的研究成果,已经审阅过论文的全部内容。

  \vspace{20pt} % 两行 三号字体 1.25 倍行距
  \begin{center}
    论文指导教师签名:
  \end{center}

  \hfill  年 \hspace{1cm} 月 \hspace{1cm} 日

  \thispagestyle{empty}

  \pagenumbering{Roman} % 页脚页码格式 - 大写罗马字母
  %%%%%%%%%%%%%%%%%%%   Abstract and Keywords  %%%%%%%%%%%%%%%%%%%%%%%
  % \clearpage
  % 摘要前关闭页码
  \pagenumbering{gobble}
  
  \markboth{摘~要}{摘~要}
  \pdfbookmark[0]{摘~~要}{cabstract}
  \chapter*{\textbf{\centering\erhao\songti 摘\quad 要}}
  \setcounter{page}{1}                                 % 单独从 1 开始编页码

  \fontsize{12pt}{20pt}\selectfont % 小四号,20pt 固定行距
  \@cabstract

  \vspace{\baselineskip}

  \noindent\textbf{\songti\sihao 关键词:} \@ckeywords

  \thispagestyle{empty} % 去掉中文摘要页眉页脚

  %%%%%%%%%%%%%%%%%%%   English Abstract  %%%%%%%%%%%%%%%%%%%%%%%%%%%%%%
  % \clearpage
  %\phantomsection
  
  \markboth{ABSTRACT}{ABSTRACT}
  \pdfbookmark[0]{ABSTRACT}{eabstract}
  \chapter*{\centering\erhao{\bf{ABSTRACT}}}

  \fontsize{12pt}{15pt}\selectfont % 小四号,20pt 固定行距

  \@eabstract
  \vspace{\baselineskip}

  % \hangafter=1\hangindent=60pt
  \noindent\textbf{\sihao KEYWORDS:} \@ekeywords

  % \thispagestyle{empty} % 去掉英文摘要页眉页脚
  % 摘要后重新开启页码
  \pagenumbering{arabic}
  \thispagestyle{empty} % 去掉英文摘要页眉页脚
}

\makeatother

%%%%%%%%%%%%%%%%%%%% 嵌入单张图片宏 %%%%%%%%%%%%%%%%%%%%
\newcommand{\figuremacro}[5]{
  \begin{figure}[#1]
    \centering
    \includegraphics[width=#5\columnwidth]{#2}
    \caption[#3]{\textbf{#3}#4}
    \label{fig:#2}
  \end{figure}
}
%% \figuremacro{ht}{Method1}{Illustration 1}{ - The Colored Red and Light blue patterns are moved from top to bottom and left to right.}{1.0}

%%%%%%%%%%%%%%%%%%%%%%%%%%% Hack %%%%%%%%%%%%%%%%%%%%
\makeatletter
\patchcmd{\src@startchapter}{\if@openright\cleardoublepage\else\clearpage\fi}{}{}{}
\makeatother

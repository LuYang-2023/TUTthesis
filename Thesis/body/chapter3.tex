% \addcontentsline{toc}{chapter}{结\quad 论} %添加到目录中
% \chapter*{结\quad 论}

\chapter{自己的工作}

\section{引言}

哈哈哈哈,我是直接把小论文的东西全部拿过来了。



\section{模型架构设计}



\subsection{模型的第一个部分}

% \subsubsection{CBS模块}

% CBS 模块由普通卷积、批归一化(Batch Normalization)和 SiLU 激活函数组成,用于从输入图像中提取更优质的特征。
\subsubsection{第一个部分的第一个小创新}





\subsubsection{第一个部分的第二个小创新}



\subsection{模型的第二个部分}



\subsection{模型的第三个部分}



\section{实验结果与分析}

\subsection{实验设置}
实验在配备Intel(R) Core(TM) i9-10900K处理器和Nvidia Geforce RTX 3090显卡的实验平台上进行。软件环境为CUDA 11.8和Python 3.8.5,实验框架采用Pytorch,具体配置见表~\ref{system_config_table}(PS:这是表格)。

\begin{table}[!htbp]
  \centering
  \caption{\hspace{-1em}实验硬件和软件配置}
  \label{system_config_table}
  \vspace{0.5em}
  \fontsize{10}{12}\selectfont
  \begin{tabular}{cc}
    \toprule
    \textbf{硬件或软件} & \textbf{参数} \\
    \midrule
    处理器(CPU) & Intel(R) Core(TM) i9-10900K \\
    显卡(GPU) & Nvidia Geforce RTX 3090 \\
    显存 & 24GB \\
    深度学习框架 & Pytorch 1.12.1 \\
    GPU加速环境 & CUDA 11.8 \\
    编程语言 & Python 3.8 \\
    \bottomrule
  \end{tabular}
\end{table}

为确保实验对比的公平性,所有实验均采用相同的超参数配置和数据增强方法。具体而言,训练过程中使用Adam优化器,Batch size设置为16,初始学习率为0.001,训练Epoch设置为300次,并采用Mosaic数据增强技术。


\subsection{实验结果对比}



\subsection{消融实验}



\section{本章小结}


本章提出了一种XXXXXX,解决了什么问题,在夸一夸自己的模型(具有较高的应用价值)。